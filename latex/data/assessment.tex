{
  \setlength{\parindent}{0em}
  \linespread{1}

  \vspace*{-2.1em}

  {
    \centering
    \songti\xiaoer\bfseries
    毕~~业~~论~~文~~(设~~计)~~考~~核 \par
  }

  \vspace{1.1em}

  {
    \songti\sihao\bfseries
    一、\; 指导教师对毕业论文(设计)的评语 \par
  }

  跨域推荐是当前的研究热点,作者以该方向为研究目标,选题具有较高的应用价值。作者在对推荐算法的国内外研究现状进行深入分 析的基础上,提出了结合标签主题的跨域推荐方法。通过在 MovieLens 和 LibraryThings 数据集上开展的实验分析表明,该方法具有一定的先 进性和较高的实用性。论文工作表明作者具有扎实的基础理论知识, 以及一定的科研工作的能力。论文条理清楚,层次分明,达到了本科毕业论文的要求。

  \vspace{8em}

  {
    \songti\xiaosi\bfseries
    \hfill 指导教师(签名) \; \underline{\hspace{5em}}

    \vspace{0.1em}

    \hfill \hspace{2em} 年 \hspace{1em} 月 \hspace{1em} 日 \par
  }

  \vspace{0.7em}

  {
    \songti\sihao\bfseries
    二、 \; 答辩小组对毕业论文(设计)的答辩评语及总评成绩:
  }

  \vspace{14.7em}

  {
    \renewcommand{\arraystretch}{1.5}
    \songti\xiaosi\bfseries
    \hfill \begin{tabular}{|c|m{4.1em}|m{4.1em}|m{4.1em}|m{9.1em}|c|}
      \hline
      成绩比例 & {\centering 开题报告 \\ 占(20\%)} & {\centering 外文翻译 \\ 占(10\%)} & {\centering 文献综述 \\ 占(10\%) } & {\centering 毕业论文(设计) \\ 质量及答辩占(60\%)} & 总成绩 \\
      \hline
      分值 & & & & & \\
      \hline
    \end{tabular} \par
  }

  \vspace{2em}

  {
    \songti\xiaosi\bfseries
    \hfill 答辩小组负责人(签名) \; \underline{\hspace{5em}}

    \vspace{0.1em}

    \hfill \hspace{2em} 年 \hspace{1em} 月 \hspace{1em} 日 \par
  }
}
