%!TEX root = ../main.tex
\chapter{总结和展望}

\section{本文工作总结}
在本文中,我们首先梳理了推荐算法的研究现状,并介绍了推荐系统中的数据稀疏和冷启动问题。为了缓解这两个问题,我们提出了一个结合标签主题的跨域推荐模型,在这个模型中,我们使用主题建模采集物品标签中的语义信息,并将语义信息引入了基于矩阵分解的协同过滤模型,然后以标签主题作为跨域传递的信息,将模型扩展到跨域的场景。最后通过对 MovieLens 和 LibraryThing 数据集进行的一系列实验,证明了该模型可以利用额外的标签信息缓解数据稀疏和冷启动问题,并且可以在没有共享用户的情况下,利用辅助域的评分和标签信息提升目标域的评分预测效果。
本文工作总结如下:
\begin{enumerate}
\item 对推荐系统的背景和研究现状进行了介绍,并阐述了目前研究中所面临的困难与挑战。
\item 对本文的相关研究成果进行了综述,详细阐述了经典的协同过滤模型,以及主题建模和跨域推荐的思想。
\item 将标签信息结合到传统的矩阵分解模型,并利用主题建模采集了标签的语义信息,使得模型可以很好地处理数据稀疏和冷启动问题,之后将模型扩展到多个域上,利用辅助域的标签和评分信息提升了目标域的评分预测效果。
\item 阐述了模型参数的设置方法,在两个真实的数据集上对模型进行了评估,并与其它的模型进行了对比实验。
\end{enumerate}

\section{未来工作展望}
本文将主题建模和跨域推荐的思想结合到传统的矩阵分解模型中,充分利用标签这种隐式反馈信息缓解了数据稀疏和冷启动问题,从而获得了更好的推荐效果。同时,有几个方面值得未来的继续探索:
\begin{enumerate}
\item 本文使用 LDA 主题建模的方式对标签进行聚类,去除冗余信息提升了算法效率,并且扩充了标签的含义,提升了在数据稀疏情况下的预测效果。那么,我们可以尝试使用其它的标签聚类方法,例如比较流行的基于标签共现的标签聚类算法\cite{王娅丹2015标签共现的标签聚类算法研究},并对比采用不同聚类算法时模型的预测效果。
\item 我们认为物品的标签集合反应了物品的特征,因此提出的模型主要针对物品的标签集进行信息挖掘。另一方面,我们可以尝试挖掘用户标签集中的信息,因为用户通常具有某些方面的偏好,而这些偏好会反应在用户标注的标签上。
\item 本文提出的模型可以适应多个域的情况,只要这些域中有重叠的标签集合。
因此可以尝试在多个数据集上对模型进行测试,以充分利用不同域之间的关联性。
\end{enumerate}

