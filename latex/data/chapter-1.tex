%!TEX root = ../main.tex
\chapter{绪论}

\section{课题背景}

随着信息技术和互联网的发展,人们逐渐从信息匮乏的时代走入了信息过载的时代,海量信息的复杂性和不均匀性使得信息获取变得困难而耗时,无论信息消费者还是信息生产者都遇到了很大的挑战。关于信息过载的问题,代表性的解决方案是分类目录和搜索引擎\cite{项亮2012推荐系统实践},分类目录只能覆盖少量内容而越来越不能满足用户的需求,搜索引擎可以让用户通过搜索关键词找到自己需要的信息,但是,搜索引擎需要用户主动对需求提供描述,当用户无法找到准确描述自己需求的关键词时,搜索引擎就无能为力了。

个性化推荐系统根据用户过往的行为,分析用户的兴趣模式,自动为用户过滤掉低相关的内容,呈现符合用户品味的个性化的产品建议,大大降低了用户检索信息的成本。与搜索引擎不同,推荐系统不需要用户提供明确的需求,从某种意义上来说,推荐系统和搜索引擎是两个互补的技术,推荐系统满足了用户有明确目的时的主动查找需求,而推荐系统能够在没有明确目的时帮助用户发现感兴趣的内容。

应用个性化推荐技术需要两个条件,第一个是存在信息过载的情况,第二个是用户在大部分时候没有明确的需求。越来越多的网站成功地引入了推荐系统,广泛利用推荐系统的领域包括电子商务、电影和视频、音乐、社交网络、基于位置的服务、个性化广告等。

推荐系统可关联到很多相关研究领域,例如用户建模、机器学习和信息检索等 \cite{fernandez2012cross} 。由于其与日俱增的重要性,它在 20 世纪 90 年代发展成一个独立的研究领域 \cite{肖力涛2016基于隐式因子和隐式主题的跨域推荐算法研究}。在推荐的过程中,推荐的准确性,以及推荐算法的效率等问题就是推荐算法研究的着重点。

推荐系统依赖于不同类型的用户行为数据,最理想的是高质量的显式反馈行为,即用户对物品兴趣的明确输入,主要的方式就是评分。通常,显式反馈产生稀疏的偏好度矩阵。与之相对应的是隐式反馈行为,即那些不能明确反应用户喜好的行为,例如购买商品、浏览页面、评论或甚至鼠标移动。相比显式反馈,隐式反馈虽然不明确,但数据量更大,因此可以利用隐式反馈缓解数据稀疏性问题。

推荐系统根据用户的历史行为预测未来的行为和兴趣,因此大量的用户数据是实现推荐系统的前提。用户物品的偏好度矩阵通常是非常稀疏的,因为单个用户浏览或使用过的物品只是很小的一部分,这样的稀疏矩阵导致潜在的关联度降低,影响推荐算法对用户兴趣的建模。另外,对于系统中加入的新用户或新物品,因为没有他们的历史数据,就无法为其关联其它用户或物品,这就产生了冷启动问题。如何克服冷启动和数据稀疏性问题是目前研究面临的主要挑战。


\section{本文的主要工作}
为了缓解上面提到的冷启动问题和数据稀疏性问题,本文提出了一种结合标签主题的跨域推荐算法(Cross-domain Collaborative Filtering with Tag Topics)。

目前,一个研究的趋势是在协同过滤中混合主题模型来处理文本内容信息,例如,Wang 和 Blei 提出了一个协同主题回归(CTR)\cite{Wang2011Collaborative}模型用于学术文章的推荐,在冷启动问题上取得了良好的效果。

跨域推荐尝试利用辅助域中的信息来协助目标域上的推荐,现有的跨域方法大都要求不同域之间存在共享用户,我们希望找到在没有共享用户的情况下关联多个域的方法,利用跨域推荐提高推荐系统的效果。

在本文的研究中,我们依赖于不同域中的重叠的标签词汇,将标签作为连接不同域的桥梁,例如,标签“romantic”可以用于描述一个电影,也可以是一首歌曲或是一处风景。因此,目标域可以学习到辅助域中的某个标签对评分的影响,即,如果在辅助域中的某个标签存在时,关联的评分通常较高,则可以将这种依赖关系从辅助域传递到目标域。

本文提出的模型在传统的矩阵分解模型的基础上,引入标签这种隐式反馈信息,使用主题建模采集标签中的语义信息,提高单一域上的推荐效果。之后,我们将跨域推荐的思想结合进来,利用辅助域中丰富的标签信息,使得模型在较为稀疏的数据集上具有良好的表现。

为了评估我们提出的模型的效果,我们使用两个真实的数据集进行了一系列实验,并将结果与现有的推荐算法进行了比较,说明结合标签信息对于提升推荐效果具有帮助。


\section{本文结构安排}
本文分为五个章节:

第一章主要介绍了本文的课题背景,并针对传统模型的问题进行了分析,提出本文的主要内容和研究思路。

第二章是相关技术综述,首先在整体上介绍了推荐系统的研究现状,之后详细介绍了基于协同过滤的推荐算法、LDA 主题模型以及跨域推荐的思想。

第三章提出了本文的结合标签主题的跨域推荐算法,并说明了模型参数的学习方法。

第四章从数据集的预处理、评价标准以及参数的选取等方面对实验进行了介绍,将提出的模型与现有的算法进行比较,论证本文模型的有效性。

第五章总结了本文的主要工作,并提出未来可以改进的方向。



